% 和欧混植のため,ptex を想定しています。
\def\ExerciseNumber{1.1}

\font\ninebf=cmbx9
\setbox0=\hbox{\ninebf COMMENT \ExerciseNumber}
\newdimen\enviromentspace
\enviromentspace=\wd0

\def\Title#1{\noindent{\ninebf \hbox to\enviromentspace{#1 \ExerciseNumber}}}



\Title{E\hfil X\hfil E\hfil R\hfil C\hfil I\hfil S\hfil E}
%After you have mastered the material in this book, what will you be: a \TeX pert, or a \TeX nician?
本書の題材をすべて習得したら、あなたは\TeX pert になっているだろうか?或いは\TeX nicianだろうか?

\Title{A\hfil N\hfil S\hfil W\hfil E\hfil R}
\TeX nician である(専門性の割に不当に安く使われることがままあるが)。
しばしば\TeX acker とも呼ばれる。

\Title{C\hfil O\hfil M\hfil M\hfil E\hfil N\hfil T}
\TeX は``technology''の``tech-''から取っているので,technician をもじった\TeX pert が素直。
\TeX acker とも呼ばれると書かれているが,これは\TeX に hacker を加えたものだろう。\end